\subsection{Markov Logic networks (MLNs)}
Markov Logic networks \cite{markovLogicNetworks} are defined as a collection of first order logical formulas \(F_j\) (for \(j\in 1\ldots K\)), each associated with a real number \(w_j\), the {\em weight} of the rule. Assume given a set of constants \(\{c_1,\ldots, c_L\}\) (in a 1-1 relationship with intended domain of interpretation), and \(a_1,\ldots, a_N\), an enumeration of the ground atoms (built from the given constants and the predicates in the formulas). Then the joint probability of the network given the truth values \(X_1, \ldots, X_n\) (for the ground atoms \(a_i\)) is given by:
\[ joint(X_1,\ldots, X_n)= (1/Z)\times \prod_{F_j} e^{w_j n_j}
\]
\noindent where \(n_j\) is the number of true groundings of \(F_j\) for the given assignment. \(Z\) is the normalization factor.

We can represent this MLN as a PCC agent \mcode{joint(X1,...,Xn)} using an interpreter for quantifier-free first-order formulas written in CCP (see the agent {\tt v/2}). 
For any formula \(F_i\) let \(F_i^1, \ldots, F_i^{p_i}\) enumerate its \(p_i\) groundings. For any ground formula \(F\), let \([F]\) stand for the formula obtained by replacing each occurrence of \(a_k\) by the variable \(X_k\) (for all \(k\)). 

The predicate \mcode{joint/n} has a single rule with \(n+2\times\Pi_{j\in 1\ldots K} p_j\) atoms in the body. The \(n\) \mcode{b(Xi)} agents each probabilistically guess the value of \mcode{Xi} (drawn from the domain \mcode{\{true,false\}}. Thus computation will result in one probabilistic branch for each of these valuations, each with a probability factor of \(0.5^{n}\) contributed by the \mcode{b/1} agents (hence these factors will get normalized out). The subsquent pairs of agents force a a value for \(\mbox{\tt Y}_{\mbox{\tt i}}^{\mbox{\tt j}}\) (for {\tt i} ranging from {\tt 1} to {\tt K} and for {\tt j} ranging from {\tt 1} to \(\mbox{\tt p}_{\mbox{\tt i}}\) ) by evaluating the formula \(\mbox{\tt [F}_{\mbox{\tt i}}^{\mbox{\tt j}}\mbox{\tt ]}\) (for the given valuation provided by \(\mbox{\tt X}_{\mbox{\tt i}}\)), and contribute a probability factor for this choice. 
\begin{lstlisting}[mathescape=true]
joint(X$_1,\ldots, $X$_n$) ->
  b(X$_1$), ..., b(X$_n$),
  Y$_1^1 \sim $ true/exp(w$_1$)+ false/1, v([F$_1^1$],Y$^1_1$),
  ...,
  Y$_1^{p_1}\sim$ true/exp(w$_1$)+ false/1 , v([F$_1^{p_1}$],Y$_{p_1}^1$),
  ...,
  Y$_K^1\sim$ true/exp(w$_K$)+ false/1,  v([F$_K^1$],Y$_K^1$)
  ...,
  Y$_K^{p_K}\sim$ true/exp(w$_K$)+ false/1,  v([F$_K^{p_k}$],Y$_K^{p_K}$),
	
b(X) -> X$\sim$true/0.5 + false/0.5.
\end{lstlisting}
\noindent Note that the variables \(\mbox{\tt Y}_{\mbox{\tt i}}\) are local to the body (existentially quantified).

The valuation predicate {\tt v/2} is straightforward to define -- it simply evaluates the first argument based on the values of the \(\mbox{\tt X}_{\mbox{\tt i}}\) supplied at the leaves, using standard CCP idioms.
\begin{lstlisting}[mathescape=true]
v(and(A,B),V) -> v(A,VA), v(B,VB), and(VA,VB,V).
v(or(A,B),V) -> v(A,VA), v(B,VB), or(VA,VB,V).
v(not(A),V) -> v(A, VA), not(VA,V).
v(v(X),V) -> X=V.
and(false, \_, Y) -> Y=false.
and(\_, false, Y) -> Y=false.
and(true,true,Y) -> Y=true.
or(true, \_, Y) -> Y=true.
or(\_, true, Y) -> Y=true.
or(false,false,Y) -> Y=false.
not(true,Y) -> Y=false.
not(false,Y)->Y=true.
\end{lstlisting}

\begin{example}\label{ex_mln}
Following the example given by Richardson et al. \cite{markovLogicNetworks}, we consider the following three formulas $F_1$, $F_2$ and $F_3$: \emph{``Smoking causes cancer''} $F_1=\neg sm(X) \vee ca(X)$ with weight $w_1=1.5$; 
\emph{``If two people are friends, either both smoke or neither does''} $F_2 = \neg fr(X, Y) \vee sm(X) \vee \neg sm(Y)$ and $F_3= \neg fr(X, Y) \vee \neg sm(X) \vee sm(Y) $ with weights $w_2=w_3=1.1$.
\begin{figure}[h!]
\centering
\begin{tikzpicture}[scale=0.53]
\scriptsize
\tikzstyle{every node}=[draw,shape=ellipse,fill=gray!30];
\node (x1) at (0, 2) {$fr(a,a)$};
\node (x2) at (2, 0) {$ca(a)$};
\node (x3) at (4, 2) {$sm(a)$};
\node (x4) at (6, 0) {$fr(b,a)$};
\node (x5) at (6, 4) {$fr(a,b)$};
\node (x6) at (8, 2) {$sm(b)$};
\node (x7) at (10, 0) {$ca(b)$};
\node (x8) at (12, 2) {$fr(b,b)$};
\draw [-]  (x1) -- (x3);
\draw [-]  (x2) -- (x3);
\draw [-]  (x3) -- (x4);
\draw [-]  (x3) -- (x5);
\draw [-]  (x3) -- (x6);
\draw [-]  (x6) -- (x8);
\draw [-]  (x6) -- (x7);
\draw [-]  (x6) -- (x4);
\draw [-]  (x6) -- (x5);
%\coordinate [label=right:$c_1$] (p) at (5.5,2.8);
%\coordinate [label=right:$c_2$] (p) at (5.5,1.3);
%\coordinate [label=right:$c_3$] (p) at (1.6,2.3);
%\coordinate [label=right:$c_4$] (p) at (2.2,1.2);
%\coordinate [label=right:$c_6$] (p) at (8.6,1.2);
%\coordinate [label=right:$c_5$] (p) at (9.3,2.3);
\end{tikzpicture}
\caption{MLN of Example \ref{ex_mln}}
\label{mln}
\end{figure}
In Figure \ref{mln} we can see the graph of the ground Markov network defined by formulas $F_1$, $F_2$ and $F_3$ and the constants \emph{Anna} ($a$) and \emph{Bob} ($b$).


The equivalent PCC agent is defined as follows. Assume the ground atomic formulas are enumerated as:
\lstinline|fr(a,b)|,
\lstinline|fr(a,a)|,
\lstinline|sm(a)|,
\lstinline|sm(b)|,
\lstinline|fr(b,b)|,
\lstinline|ca(a)|,
\lstinline|fr(b,a)| and
\lstinline|ca(b)|. Then we have:
\begin{lstlisting}[mathescape=true]
joint(X$_1$,...,X$_8$) $\rightarrow$ 
  b(X$_1$), ... , b(X$_8$),
  Y$_1^1$$\sim$ true/e$^{w_1}$+ false/1, 
  v(or(not(X$_3$), X$_6$),Y$_1^1$),
  Y$_1^2$$\sim$ true/e$^{w_1}$+ false/1, 
  v(or(not(X$_4$), X$_8$),Y$_1^2$),	
  Y$_2^1$$\sim$ true/e$^{w_2}$+ false/1,
  v(or(not(X$_1$),or(X$_3$, not(X$_4$)),Y$_2^1$),
  Y$_2^2$$\sim$ true/e$^{w_2}$+ false/1,
  v(or(not(X$_2$),or(X$_3$, not(X$_3$)),Y$_2^2$),
  Y$_2^3$$\sim$ true/e$^{w_2}$+ false/1,
  v(or(not(X$_5$),or(X$_4$, not(X$_4$)),Y$_2^3$),
  Y$_2^4$$\sim$ true/e$^{w_2}$+ false/1,
  v( or(not(X$_7$),or(X$_4$, not(X$_3$)),Y$_2^4$),
  Y$_3^1$$\sim$ true/e$^{w_3}$+ false/1,
  v(or(not(X$_1$),or(not(X$_3$), X$_4$)),Y$_3^1$),
  Y$_3^2$$\sim$ true/e$^{w_3}$+ false/1,
  v(or(not(X$_2$),or(not(X$_3$), X$_3$)),Y$_3^2$),
  Y$_3^3$$\sim$ true/e$^{w_3}$+ false/1,
  v(or(not(X$_5$),or(not(X$_4$), X$_4$)),Y$_3^3$),	
  Y$_3^4$$\sim$ true/e$^{w_3}$+ false/1, 
  v(or(not(X$_7$),or(not(X$_4$), X$_3$)),Y$_3^4$).
\end{lstlisting}
\end{example}


It is important to notice that the PCC representation of a MLN has un-normalised weights for its random variables. It is easy to find a normalisation constant that allow to recover the original MLN probability distribution over complete truth assignment of the variables (corresponding to successful executions of the PCC agent) since in the PCC formulation the probability of an execution is the product of the probability over the full set of variables (in each execution we sample all the random variables). So we consider a random variable of the form:
\begin{lstlisting}[mathescape=true]
Y$_i^j$$\sim$ true/e$^{w_i}$+false/e$^0$
\end{lstlisting}
as:
\begin{lstlisting}[mathescape=true]
Y$_i^j$ $\sim$ true/$\frac{e^{w_i}}{e^{w_i}+e^0 }$+false/$\frac{e^0}{e^{w_i}+e^0 }$
\end{lstlisting}
 and at the end of the computation we multiply every execution probability for the following normalisation constant:
 $$Z=\prod_{i=1}^k \prod_{j=1}^{p_i} (e^{w_i} +e^0)=\prod_{i=1}^k (e^{w_i} +e^0)^{p_i}$$






\begin{theorem}
Given a MLN $\mathcal{M}$ defined over a set of clauses $F_j$ and its PCC translation $\mathcal{P}$ 
the probability distribution $pd_{\mathcal{M}}$ defined by $\mathcal{M}$ over the complete truth assignments of the ground atoms  is equal to the probability distribution $pd_{\mathcal{P}}$ defined by $\mathcal{P}$ over the complete truth assignments of the ground atoms.
\end{theorem}
\begin{proof}
Given a input $x_1, \ldots ,x_n$ where $x_1, \ldots , x_n$ are the truth values of the ground atoms $a_1,\ldots, a_n$ generated by the MLN, we want to prove that the probability of $pd_{\mathcal{P}}$ is equal to $pd_{\mathcal{M}}$ on this input. Given the definition of MLN, we have that $pd_{\mathcal{M}}(x_1, \ldots ,x_n)=\prod_{F_j} e^{w_j n_j}$.

Each execution of the PCC agent provide a complete instantiation of the variables $Y_i^j$ by definition of the predicate $jt$. We have thus that the truth value of each grounding formula $[F]_i^j$ is defined: \lstinline[mathescape=true]{v([F]$_i^j$,Y$_i^j$)}. 

The probability of such combination (and the corresponding execution) is the product of the factors associated with each choice: we have a factor $e^{w_i}$ for each \lstinline[mathescape=true]{v([F]$_i^j$,true)} used and $1$ for each \lstinline[mathescape=true]{v([F]$_i^j$,false)} used. Only one of these combinations will succeed, since the input $x_1, \ldots ,x_n$ determines uniquely only one consistent instantiation of the $[F]_i^j$. 
Thus there is only one sampling execution that will succeeds, since every input $x_1,\ldots,x_N$ is consistent with only one instantiation of the $[F]_i^j$.
\end{proof}

%We can define a the same encoding of MLNs in PCCs also in SLPs with minor modifications that we omit due to lack of space. It is important to notice that the SLP formalisation of MLN corresponds to an un-normalised and impure SLP program: un-normalised because the sum of the weights for clauses whose heads share the same predicate could be greater than $1$, and impure since there are rules that don't have a weight.

